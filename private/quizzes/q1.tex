\documentclass{article}
\usepackage[margin=1.5cm,bottom=2cm]{geometry}
\usepackage{fancyhdr}
\usepackage{graphicx}
\pagestyle{fancy}
\usepackage{amsmath}

\begin{document}
\fancyhead[L]{ \includegraphics[width=2cm]{/Users/tjwilli/google_drive/course_materials/global/au_logo.png} }
\fancyhead[R]{PHYS 2250: General Physics II}
\fancyfoot[C]{\thepage}
\vspace*{0cm}
\begin{center}
	{\LARGE \textbf{Quiz 1}}\\
	\vspace{0.25cm}
	%{\Large Vector Review (Ungraded)}\\
	\vspace{0.25cm}
	{\Large Friday, September 17}
\end{center}
The following information may or may not be of use:\\
\hrulefill\\
\begin{align*}
	\varepsilon_0 &= 8.85\times10^{-12} \mathrm{C}^2\ \mathrm{N}^{-1}\ \mathrm{m}^{-2}\\
	k&=\frac{1}{4\pi\varepsilon_0}=9\times10^9 \mathrm{N}\ \mathrm{m}^2\ \mathrm{C}^{-2}\\
	|\vec{E}_{dipole, on-axis}| &\approx \frac{1}{4\pi\varepsilon_0}\frac{2p}{r^3}\\
	|\vec{E}_{dipole, perp}| &\approx \frac{1}{4\pi\varepsilon_0}\frac{p}{r^3}
\end{align*}

\hrulefill \\
\\
\begin{enumerate}
	\item In a certain coordinate system, a point charge $q_1=-4\  \mu$C is located at the position $\vec{r}_1=<4,-1,0>$ m. A second charge $q_2=6\ \mu$C sits at $\vec{r}_2=<0,5,0>$ m. Finally, a third charge $q_3=9\ \mu$C is at $\vec{r}_3=<-3,-7,0>$ m. What is the net force exerted on $q_3$ due to $q_1$ and $q_2$? Be sure to express your answer as a vector with correct units.
\end{enumerate}

\end{document}
\documentclass{article}
\usepackage[margin=1.5cm,bottom=2cm]{geometry}
\usepackage{fancyhdr}
\usepackage{graphicx}
\pagestyle{fancy}

\begin{document}
\fancyhead[L]{ \includegraphics[width=2cm]{au_logo.png} }
\fancyhead[R]{PHYS 2250: General Physics II}
\fancyfoot[C]{\thepage}
\vspace*{0cm}
\begin{center}
	{\LARGE \textbf{Quiz 1}}
	%\vspace{0.25cm}
	%{\Large Due: Friday, September 11}
\end{center}
You may or may not make use of the following:
\vspace{.5cm}

\renewcommand{\arraystretch}{2}
\begin{tabular}{ccc}
$\epsilon_0=8.85\times10^{-12}\ Nm^2C^{-2}$ & $k=\frac{1}{4\pi\epsilon_0}=9\times10^9\ C^2N^{-1}m^{-2}$\\
$|\vec{E}_\mathrm{dipole,on-axis}| \approx \frac{1}{4\pi\epsilon_0}\frac{2p}{r^3}$ & $|\vec{E}_\mathrm{dipole,perp}| \approx \frac{1}{4\pi\epsilon_0}\frac{p}{r^3}$
\end{tabular}
%Complete the following problems from your textbook at the end of Chapter 13.
\begin{enumerate}
\item A point charge located at $<4,0>$ meters has a charge of $-10$ nC. What is the electric field vector at the location $<1,4>$ meters?
\vspace{10cm}
\item A certain molecule consists of a positive charge located at $<2\times10^{-12},0>$ meters and a negative charge located at $<-2\times10^{-12},0>$ meters. Each charge has a magnitude of $10e$ (one being negative, one being positive). What is the magnitude of the dipole moment, $p$? You may leave your answer in terms of $e$.
\end{enumerate}

\end{document}
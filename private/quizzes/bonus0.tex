\documentclass{article}
\usepackage[margin=1.5cm,bottom=2cm]{geometry}
\usepackage{fancyhdr}
\usepackage{graphicx}
\usepackage{xcolor}
\usepackage{transparent}
\pagestyle{fancy}

\begin{document}
\fancyhead[L]{ \includegraphics[width=2cm]{au_logo.png} }
\fancyhead[R]{PHYS 2250: General Physics II}
\fancyfoot[C]{\thepage}
\vspace*{0cm}
\begin{center}
	{\LARGE \textbf{Bonus Quiz 1}}
	%\vspace{0.25cm}
	%{\Large Due: Friday, September 11}
\end{center}
You may or may not make use of the following:
\vspace{.5cm}

\renewcommand{\arraystretch}{2}
\begin{tabular}{ccc}
$\epsilon_0=8.85\times10^{-12}\ Nm^2C^{-2}$ & $k=\frac{1}{4\pi\epsilon_0}=9\times10^9\ C^2N^{-1}m^{-2}$\\
$|\vec{E}_\mathrm{dipole,on-axis}| \approx \frac{1}{4\pi\epsilon_0}\frac{2p}{r^3}$ & $|\vec{E}_\mathrm{dipole,perp}| \approx \frac{1}{4\pi\epsilon_0}\frac{p}{r^3}$
\end{tabular}
%Complete the following problems from your textbook at the end of Chapter 13.
\begin{enumerate}
\item {
	Two point charges are arranged in a region of space as follows:

$q_1 = -2$ nC, located at $<-3,1>$ mm

$q_2 = -6$ nC, located at $<1, 2>$ mm

What is the electric field vector $\vec{E}$ (including units!) at the location $<0,4>$ mm?

\textit{Recall: nC = $10^{-9}$ C, mm = $10^{-3}$ m}
}

\item {
	A third charge $q_3=4$ nC is added to the above configuration and set at the location $<0,4>$ mm. What is the net electric force on this charge, due to the presence of the other two charges?
}
\end{enumerate}

\vspace{16.2cm}
\begin{flushright}
	%\color{gray}
	\transparent{0.1}
	\color{gray}
	\tiny{A neutron walks into a bar...}
\end{flushright}
\end{document}
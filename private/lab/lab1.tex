\documentclass{article}
\usepackage[margin=1.5cm,bottom=2cm]{geometry}
\usepackage{fancyhdr}
\usepackage{graphicx}
\usepackage[section]{placeins}
\pagestyle{fancy}

\begin{document}
\fancyhead[L]{ \includegraphics[width=2cm]{/Users/tjwilli/google_drive/course_materials/global/au_logo.png} }
\fancyhead[R]{PHYS 2250: General Physics II}
\fancyfoot[C]{\thepage}
\vspace*{0cm}
\begin{center}
	{\LARGE \textbf{Lab 1}}\\
	\vspace{.25cm}
	{\Large Charging}
	%\vspace{0.25cm}
	%{\Large Due: Friday, September 4}
\end{center}

\section*{Introduction}
When charging by contact, one material transfers some of its electrons to another. This leaves both objects charged, one positively and one negatively. 

In this lab, you will investigate this effect and determine the sign of the charge on each object.

\section*{Charging}
On your lab table, you have three types of cloth and three different rods. The gray rod is PVC, the clear one is glass, and the black is rubber. The furry cloth is, well, fur; the blue cloth is silk; the black is wool. This gives you nine possible rod-cloth combinations. Your job is to devise an experiment to determine for each combination which object gains electrons and which object loses them when they are rubbed together. You may use the Vernier charge measurement apparatus on only one combination, and from that you should be able to determine the sign ($+/-$) of the charge of the rest of the combinations. Record your results in a table.

\begin{table}[ht!]
	\centering
	\begin{tabular}{|c|c|c|}
		\hline
		\textbf{Rubbing material} & \textbf{Rod material} & \textbf{Result} \\
		\hline
		\textit{cotton (example)} & \textit{wood (example)} & \textit{electrons transferred from cotton to wood*}\\
		\hline
		Fur & PVC & \\
		\hline
		Silk & PVC & \\
		\hline
		Wool & PVC & \\
		\hline
		Fur & Rubber & \\
		\hline
		Silk & Rubber & \\
		\hline
		Wool & Rubber & \\
		\hline
		Fur & Glass & \\
		\hline
		Silk & Glass & \\
		\hline
		Wool & Glass & \\
		\hline
	\end{tabular}
\end{table}

{\scriptsize\textit{* I don't actually know if this is the case, it's just an example}}\\


After you have finished, check your results using the Vernier equipment (see the next section for details). Make a note of any charge measurements that are inconsistent with your previous observations.
\subsection*{How to Use the Vernier Charge Sensor}
\begin{enumerate}
	\item First you must zero the charge sensor. \textbf{This should be done before every measurement!}
	\begin{enumerate}
		\item To zero the sensor, first connect the black ground wire from the metal ground plate to the metal cage
		\item Now connect the charge sensor. Connect the black lead to the metal ground plate; connect the red lead somewhere on the metal cage.
		\item Now that the charge sensor is connected and the charge cage is grounded, press and hold the reset button on the charge sensor (not the zero button in Logger Pro)
		\item You should notice the charge reading in Logger Pro become close to zero; you have now zeroed the sensor
	\end{enumerate}
	\item Now that you have zeroed the sensor, you can use it to take data
	\item Leave the charge sensor connected as it was before, but \textbf{disconnect the black ground wire} connecting the metal plate to the metal cage
	\item With the apparatus ungrounded, simply place the object whose charge you want to measure inside of the metal bucket that is within the cage. You do not need to touch the object to the bucket, just hold it inside.
	\item As long as you hold the object inside of the bucket, the object's charge will display on Logger Pro.
\end{enumerate}

\section*{Questions}
\begin{enumerate}
	\item Explain any discrepancies between your predicted and measured results in part 1. Be sure to discuss possible sources of errors and uncertainties (be more specific than ``human error''!)
	\item Can a charged object ever attract a neutral object? Explain.
	\item Can a charged object ever repel a neutral object? Explain.
\end{enumerate}
\end{document}
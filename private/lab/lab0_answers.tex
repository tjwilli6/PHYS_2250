\documentclass{article}
\usepackage[margin=1.5cm,bottom=2cm]{geometry}
\usepackage{fancyhdr}
\usepackage{graphicx}
\usepackage[section]{placeins}
\pagestyle{fancy}
\usepackage{xcolor}
\usepackage{hyperref}
\usepackage{graphicx}

\hypersetup{colorlinks=true,urlcolor=blue,urlbordercolor=blue}
\begin{document}
\fancyhead[L]{ \includegraphics[width=2cm]{/Users/tjwilli/google_drive/course_materials/global/au_logo.png} }
\fancyhead[R]{PHYS 2250: General Physics II}
\fancyfoot[C]{\thepage}
\vspace*{0cm}
\begin{center}
	{\LARGE \textbf{Lab 0}}\\
	\vspace{.25cm}
	{\Large Measurements and Uncertainty}
	%\vspace{0.25cm}
	%{\Large Due: Friday, September 4}
\end{center}

You are running an experiment to measure the velocity of a cart rolling down a track.You first measure the mass of the cart 5 different times:

\begin{center}
	\begin{tabular}{|c|c|}
		\hline
		Measurement & Mass \\
		\hline
		1 & 251.05 grams \\
		\hline
		2 & 249.99 grams \\
		\hline
		3 & 251.07 grams \\
		\hline
		4 & 251.05 grams \\
		\hline 
		5 & 251.14 grams \\
		\hline
	\end{tabular}
\end{center}

Next, you measure the speed of the cart at the bottom of the track:

\begin{center}
	\begin{tabular}{|c|c|}
		\hline
		Measurement & Speed \\
		\hline
		1 & 4.2 m/s \\
		\hline
		2 & 3.9 m/s \\
		\hline
		3 & 4.1 m/s\\
		\hline
		4 & 4.0 m/s\\
		\hline 
		5 & 4.1 m/s\\
		\hline
	\end{tabular}
\end{center}


\begin{enumerate}
	\item What is your estimate (including uncertainty) of the mass of the cart?\\
	Answer: I used \href{https://docs.google.com/spreadsheets/d/1X5Ok14Y5Rkm_orYHjFNCjtL3NYZ63yXkAczzOlAsNw4/edit?usp=sharing}{this Google spreadsheet} to find the average and standard deviation of the mass measurements. I found $m=250.86\pm 0.49$ g $= 0.25086\pm 0.00049$ kg
	\item What is your estimate (including uncertainty) of the speed of the cart?\\
	I used the same spreadsheet to calculate the speed. I find $v=4.1\pm0.1$ m/s.
	\item Suppose your theoretical prediction for the velocity is 4.12 m/s. Is your measurement consistent with this prediction?\\
	My measured velocity is $v=4.1\pm0.1$ m/s, which means the true value could be anywhere from 4 to 4.2 m/s. Since 4.12 m/s is within this interval, my measurement and prediction are consistent.
	\item Suppose your theoretical prediction for the momentum is 1.15 kg m/s. Is your measured value of the momentum consistent with this prediction?\\
	First, I must calculate the momentum, $p=mv=0.25086\cdot 4.1=1.03$ kg m/s. According to my slides from lab: 

		$\frac{\sigma_p}{p} = \sqrt{\left(\frac{\sigma_m}{m}\right)^2+\left(\frac{\sigma_v}{v}\right)^2}=\sqrt{\left(\frac{0.00049}{0.25086}\right)^2+\left(\frac{0.1}{4.1}\right)^2}=0.024$\\
		This leaves me with:\\
		$\frac{\sigma_p}{p}=0.024$\\
		I know $p=mv=0.25086\cdot 4.1=1.03$ kg m/s; I want $\sigma_p$\\
		$\frac{\sigma_p}{p}=0.024\rightarrow \sigma_p=0.024\cdot p = 0.024\cdot 1.03=0.025$ kg m/s\\
		
		So the estimate for momentum, including uncertainty, is $p=1.03\pm 0.025$ kg m/s.\\
		Is this estimate consistent with the predicted value, 1.15? No it is not, since 1.15 does not fall between 1.03-0.025 and 1.03+0.025.
	\item Suppose your lab partner runs 5 more trials and finds a speed of $3.97\pm0.09$ m/s. Is their measurement consistent with your own?\\
	The two measurements are consistent, since their uncertainty estimates overlap.
\end{enumerate}

\end{document}
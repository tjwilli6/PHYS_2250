\documentclass{article}
\usepackage[margin=1.5cm,bottom=2cm]{geometry}
\usepackage{fancyhdr}
\usepackage{graphicx}
\pagestyle{fancy}

\begin{document}
\fancyhead[L]{ \includegraphics[width=2cm]{au_logo.png} }
\fancyhead[R]{PHYS 2250: General Physics II}
\fancyfoot[C]{\thepage}
\vspace*{0cm}
\begin{center}
	{\LARGE \textbf{Lab 1}}\\
	{\Large Charging and polarization}
	%\vspace{0.25cm}
	%{\Large Due: Friday, September 4}
\end{center}

\section*{Introduction}
When charging by contact, one material transfers some of its electrons to another. This leaves both objects charged, one positively and one negatively. 

In the first part of this lab, you will investigate this effect and determine the sign of the charge on each object.

In the second part, you will quantitatively measure the effect of polarization due to an external electric field.

\section*{Part 1: Charging}
On your lab table, you have three types of cloth and three different rods. This gives you nine possible rod-cloth combinations. Your job is to devise an experiment to determine for each combination which object gains electrons and which object loses them when they are rubbed together. You may use the Vernier charge measurement apparatus on only one combination, and from that you should be able to determine the sign ($+/-$) of the charge of the rest of the combinations. Record your results in a table.

After you have finished, check your results using the Vernier equipment. Record these results in a different table.

\section*{Part 2: Polarization}
In class, we learned about how an uncharged object may experience a force in the presence of an electric field due to polarization. When a material is placed within an electric field, it forms a dipole with dipole moment $\vec{p}=\alpha\vec{E}$. Today, we will obtain an estimate of the so-called ``polarizability'' $\alpha$. 

\subsection*{Procedure}
In this procedure, you will devise an experiment to measure $\alpha$. Recall that, for a neutral object polarized by an external field:
\begin{equation}
|\vec{E}|=\left(\frac{1}{4\pi\epsilon_0}\right)^2\frac{2\alpha q_1}{r^5}
\end{equation}

and therefore the mutual force is:

\begin{equation}
|\vec{F}|=q_1|\vec{E}|=\left(\frac{1}{4\pi\epsilon_0}\right)^2\frac{2\alpha q_1^2}{r^5}
\end{equation}
Therefore, if you can measure $F$, $q_1$, and $r$, you can solve for $\alpha$.

Your job is to devise an experiment in which you measure $F$, $q_1$, and $r$ and arrive at an estimate for $\alpha$. You will want to make use of the following:
\begin{itemize}
	\item Digital balance
	\item Charged rod (choose the highest charge combination from part 1)
	\item Vernier charge measurement device
	\item Caliper
\end{itemize}

Run the experiment several times and record each value for $\alpha$.

\section*{Questions}
\begin{enumerate}
	\item Explain any discrepancies between your predicted and measured results in part 1.
	\item Can a charged object ever repel a neutral object? Explain.
	\item From part 2: what is the electric field of the charged rod at the location of the ball?
	\item What is the force experienced by the rod?
\end{enumerate}

\end{document}
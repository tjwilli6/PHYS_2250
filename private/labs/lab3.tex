\documentclass{article}
\usepackage[margin=1.5cm,bottom=2cm]{geometry}
\usepackage{fancyhdr}
\usepackage{graphicx}
\usepackage{amsmath}
\usepackage[section]{placeins}
\pagestyle{fancy}

\begin{document}
\fancyhead[L]{ \includegraphics[width=2cm]{au_logo.png} }
\fancyhead[R]{PHYS 2250: General Physics II}
\fancyfoot[C]{\thepage}
\vspace*{0cm}
\begin{center}
	{\LARGE \textbf{Lab 3}}\\
	\vspace{.5cm}
	{\Large Magnetic Field of a Solenoid}
	%\vspace{0.25cm}
	%{\Large Due: Friday, September 4}
\end{center}

\section*{Introduction}
In class, we learned the expression for the magnetic field of a loop of current: 
\begin{equation*}
	|\vec{B}|=\frac{\mu_0}{2}\frac{R^2I}{\left(z^2+R^2\right)^{3/2}}
\end{equation*}
Today we will investigate the magnetic field inside a structure of several stacked current loops: a solenoid. The magnetic field of a solenoid can be expressed as:
\begin{equation}
	|\vec{B}|=\frac{\mu_0 NI}{L}
\end{equation}
Where $I$ is the current, $N$ is the number of loops, or ``turns'' of the solenoid, and $L$ is the length of the solenoid. In this lap, we will measure $B$ as $I$ is varied in order to determine $N$.

\section*{Procedure}
The procedure for this lab is fairly straightforward:
\begin{itemize}
	\item Use the voltage knob on the power supply to vary the current
	\item Measure the magnetic field for at least 10 different values of current. Try to keep these values evenly spaced.
	\item You will need to measure the length $L$ of your solenoid.
\end{itemize}

\section*{Analysis Questions}
\begin{enumerate}
	\item You now have a table of current and field values. Using graphing software such as Excel, Google Sheets, Python, etc: make a plot of $B$ vs $I$. What does the slope of this line represent?
	\item Use this same software to fit your data to a linear curve. What is the slope? What is the intercept? What \textit{should} the intercept be?
	\item Use your value for $m$ to calculate the number of turns $N$ in the solenoid. Is your result reasonable?
\end{enumerate}

\end{document}
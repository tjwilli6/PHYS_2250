\documentclass{article}
\usepackage[margin=1.5cm,bottom=2cm]{geometry}
\usepackage{fancyhdr}
\usepackage{graphicx}
\usepackage{amsmath}
\usepackage[section]{placeins}
\pagestyle{fancy}

\begin{document}
\fancyhead[L]{ \includegraphics[width=2cm]{au_logo.png} }
\fancyhead[R]{PHYS 2250: General Physics II}
\fancyfoot[C]{\thepage}
\vspace*{0cm}
\begin{center}
	{\LARGE \textbf{Lab 6}}\\
	\vspace{.5cm}
	{\Large Circuit Analysis}
	%\vspace{0.25cm}
	%{\Large Due: Friday, September 4}
\end{center}

\section*{Introduction}
In this lab, you will use the techniques discussed in class (loop rule, node rule, Ohm's Law, etc) to analyze circuits of your own choosing.

\section*{Procedure}
You will build three circuits. For each circuit, you will first calculate and then measure the current, potential, and dissipated power associated with each circuit element.\\

\textbf{Note: before you begin, you need to measure the internal resistance of the battery, and decide if you may ignore it in your calculations or not. You may do this in any way you see fit.}

\begin{itemize}
	\item There are no restrictions for the first circuit you build.
	\item The second and third circuits must include:
	\begin{itemize}
		\item At least three resistive elements
		\item At least one junction (point where the circuit branches)
	\end{itemize}
\end{itemize}
With your lab report: you will turn in the following:
\begin{itemize}
	\item A diagram of each circuit
	\item Data tables containing the current (in amps), voltage (V) and power (W) of each resistive element
\end{itemize}
\section*{Analysis}
\begin{enumerate}
	\item What did you find to be the internal resistance of the battery? How did you measure it?
	\item For each circuit, briefly describe how you performed your calculations (write down any equations you used)
\end{enumerate}
\end{document}
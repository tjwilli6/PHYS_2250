\documentclass{article}
\usepackage[margin=1.5cm,bottom=2cm]{geometry}
\usepackage{fancyhdr}
\usepackage{graphicx}
\usepackage{amsmath}
\pagestyle{fancy}

\begin{document}
\fancyhead[L]{ \includegraphics[width=2cm]{au_logo.png} }
\fancyhead[R]{PHYS 2250: General Physics II}
\fancyfoot[C]{\thepage}
\vspace*{0cm}
\begin{center}
	{\LARGE \textbf{Exam II Study Guide}}\\
	\vspace{0.25cm}
	%{\Large Due: Friday, September 4}
\end{center}
\section*{Chapter 17}
\begin{itemize}
	\item Biot-Savart Law for point charges and currents:
	\begin{itemize}
		\item $\vec{B}=\frac{\mu_0}{4\pi}\frac{q\vec{v}\times \hat{r}}{r^2}$
		\item $\vec{B}=\int\frac{\mu_0}{4\pi}\frac{I\vec{dl}\times\hat{r}}{r^2}$
		\item $|\vec{B}|=\frac{\mu_0 I }{2\pi r}$ (very long, straight wire)
		\item $|\vec{B}|=\frac{\mu_0}{4\pi}\frac{2\pi R^2I}{\left(z^2+R^2\right)^\frac{3}{2}}$ (loop of wire)
		\item Examples: P37, P45, P46
	\end{itemize}
	You do not need to have these memorized, but you do need to know how to use them! In particular, make sure you are VERY familiar with the cross product / right hand rule.
	\item Currents
	\begin{itemize}
		\item Electron current $i=nA\bar{v}$ (you should know what each of these terms mean!)
		\item Conventional current $I=|q|i$. What are the directions of $i$ and $I$?
		\item Examples: P24, P25, P27
	\end{itemize}

	\item Bar Magnets and Dipoles
	\begin{itemize}
		\item You should be able to roughly sketch the field of a bar magnet
	\end{itemize}
\end{itemize}
\section*{Chapter 18}
\begin{itemize}
	\item Be able to explain (on a qualitative level):
	\begin{itemize}
		\item Steady state vs static equilibrium
		\item The formation of the steady-state electric field and current (i.e. the field is due to the field of the battery + the field of built of surface charge)
		\item The basic function of a battery (uses energy to maintain charge separation)
		\item Examples: P20-21
	\end{itemize}
	\item Know how to use the node rule and loop rule to solve for $E$ and $i$ everywhere throughout a circuit
	\begin{itemize}
		\item Node rule: $i_\mathrm{in} = i_\mathrm{out}$
		\item Loop rule: $\sum \Delta V = 0$ for any \underline{closed} loop.
		\item $i=nAv$, $v=uE$, $i=nAuE$
		\item In order to use the Loop rule properly, you must know the direction of $E$ everywhere in the wire, and remember how to find $\Delta V = -\vec{E}\cdot \Delta \vec{\ell}$
		\item Examples: P44, P46
	\end{itemize}
\end{itemize}
\section*{Chapter 19}
\begin{itemize}
	\item Ohm's Law and Resistors
	\begin{itemize}
		\item $I = \Delta V / R$
		\item $R = \frac{L}{\sigma A}$
		\item Equivalent resistance in series: $R_{eq} = R_1 + R_2 + R_3 + ...$
		\item Equivalent resistance in parallel: $\frac{1}{R_{eq}}=\frac{1}{R_1} + \frac{1}{R_2} + \frac{1}{R_3} + ...$
		\item Power dissipated in a resistor: $P=I\Delta V$
		\item Internal resistance of a battery: $\Delta V_\mathrm{battery} = \varepsilon - Ir_\mathrm{int}$
		\item Examples: P47, P51, P54, P56, P67, P68
	\end{itemize}
	\item Use the loop and node rule to find $\Delta V$ and $I$ over every element in a circuit
	\begin{itemize}
		\item Examples: P63 (ignore a-h, just know how to find $I$ and $\Delta V$ for each resistor), P66
	\end{itemize}
	\item Know how to work with $RC$ circuits
	\begin{itemize}
		\item Know the potential difference across a capacitor: $\Delta V= Q/C$
		\item Examples: P74, P76
	\end{itemize}
\end{itemize}
\end{document}